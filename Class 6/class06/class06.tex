% Options for packages loaded elsewhere
% Options for packages loaded elsewhere
\PassOptionsToPackage{unicode}{hyperref}
\PassOptionsToPackage{hyphens}{url}
\PassOptionsToPackage{dvipsnames,svgnames,x11names}{xcolor}
%
\documentclass[
  letterpaper,
  DIV=11,
  numbers=noendperiod]{scrartcl}
\usepackage{xcolor}
\usepackage{amsmath,amssymb}
\setcounter{secnumdepth}{-\maxdimen} % remove section numbering
\usepackage{iftex}
\ifPDFTeX
  \usepackage[T1]{fontenc}
  \usepackage[utf8]{inputenc}
  \usepackage{textcomp} % provide euro and other symbols
\else % if luatex or xetex
  \usepackage{unicode-math} % this also loads fontspec
  \defaultfontfeatures{Scale=MatchLowercase}
  \defaultfontfeatures[\rmfamily]{Ligatures=TeX,Scale=1}
\fi
\usepackage{lmodern}
\ifPDFTeX\else
  % xetex/luatex font selection
\fi
% Use upquote if available, for straight quotes in verbatim environments
\IfFileExists{upquote.sty}{\usepackage{upquote}}{}
\IfFileExists{microtype.sty}{% use microtype if available
  \usepackage[]{microtype}
  \UseMicrotypeSet[protrusion]{basicmath} % disable protrusion for tt fonts
}{}
\makeatletter
\@ifundefined{KOMAClassName}{% if non-KOMA class
  \IfFileExists{parskip.sty}{%
    \usepackage{parskip}
  }{% else
    \setlength{\parindent}{0pt}
    \setlength{\parskip}{6pt plus 2pt minus 1pt}}
}{% if KOMA class
  \KOMAoptions{parskip=half}}
\makeatother
% Make \paragraph and \subparagraph free-standing
\makeatletter
\ifx\paragraph\undefined\else
  \let\oldparagraph\paragraph
  \renewcommand{\paragraph}{
    \@ifstar
      \xxxParagraphStar
      \xxxParagraphNoStar
  }
  \newcommand{\xxxParagraphStar}[1]{\oldparagraph*{#1}\mbox{}}
  \newcommand{\xxxParagraphNoStar}[1]{\oldparagraph{#1}\mbox{}}
\fi
\ifx\subparagraph\undefined\else
  \let\oldsubparagraph\subparagraph
  \renewcommand{\subparagraph}{
    \@ifstar
      \xxxSubParagraphStar
      \xxxSubParagraphNoStar
  }
  \newcommand{\xxxSubParagraphStar}[1]{\oldsubparagraph*{#1}\mbox{}}
  \newcommand{\xxxSubParagraphNoStar}[1]{\oldsubparagraph{#1}\mbox{}}
\fi
\makeatother

\usepackage{color}
\usepackage{fancyvrb}
\newcommand{\VerbBar}{|}
\newcommand{\VERB}{\Verb[commandchars=\\\{\}]}
\DefineVerbatimEnvironment{Highlighting}{Verbatim}{commandchars=\\\{\}}
% Add ',fontsize=\small' for more characters per line
\usepackage{framed}
\definecolor{shadecolor}{RGB}{241,243,245}
\newenvironment{Shaded}{\begin{snugshade}}{\end{snugshade}}
\newcommand{\AlertTok}[1]{\textcolor[rgb]{0.68,0.00,0.00}{#1}}
\newcommand{\AnnotationTok}[1]{\textcolor[rgb]{0.37,0.37,0.37}{#1}}
\newcommand{\AttributeTok}[1]{\textcolor[rgb]{0.40,0.45,0.13}{#1}}
\newcommand{\BaseNTok}[1]{\textcolor[rgb]{0.68,0.00,0.00}{#1}}
\newcommand{\BuiltInTok}[1]{\textcolor[rgb]{0.00,0.23,0.31}{#1}}
\newcommand{\CharTok}[1]{\textcolor[rgb]{0.13,0.47,0.30}{#1}}
\newcommand{\CommentTok}[1]{\textcolor[rgb]{0.37,0.37,0.37}{#1}}
\newcommand{\CommentVarTok}[1]{\textcolor[rgb]{0.37,0.37,0.37}{\textit{#1}}}
\newcommand{\ConstantTok}[1]{\textcolor[rgb]{0.56,0.35,0.01}{#1}}
\newcommand{\ControlFlowTok}[1]{\textcolor[rgb]{0.00,0.23,0.31}{\textbf{#1}}}
\newcommand{\DataTypeTok}[1]{\textcolor[rgb]{0.68,0.00,0.00}{#1}}
\newcommand{\DecValTok}[1]{\textcolor[rgb]{0.68,0.00,0.00}{#1}}
\newcommand{\DocumentationTok}[1]{\textcolor[rgb]{0.37,0.37,0.37}{\textit{#1}}}
\newcommand{\ErrorTok}[1]{\textcolor[rgb]{0.68,0.00,0.00}{#1}}
\newcommand{\ExtensionTok}[1]{\textcolor[rgb]{0.00,0.23,0.31}{#1}}
\newcommand{\FloatTok}[1]{\textcolor[rgb]{0.68,0.00,0.00}{#1}}
\newcommand{\FunctionTok}[1]{\textcolor[rgb]{0.28,0.35,0.67}{#1}}
\newcommand{\ImportTok}[1]{\textcolor[rgb]{0.00,0.46,0.62}{#1}}
\newcommand{\InformationTok}[1]{\textcolor[rgb]{0.37,0.37,0.37}{#1}}
\newcommand{\KeywordTok}[1]{\textcolor[rgb]{0.00,0.23,0.31}{\textbf{#1}}}
\newcommand{\NormalTok}[1]{\textcolor[rgb]{0.00,0.23,0.31}{#1}}
\newcommand{\OperatorTok}[1]{\textcolor[rgb]{0.37,0.37,0.37}{#1}}
\newcommand{\OtherTok}[1]{\textcolor[rgb]{0.00,0.23,0.31}{#1}}
\newcommand{\PreprocessorTok}[1]{\textcolor[rgb]{0.68,0.00,0.00}{#1}}
\newcommand{\RegionMarkerTok}[1]{\textcolor[rgb]{0.00,0.23,0.31}{#1}}
\newcommand{\SpecialCharTok}[1]{\textcolor[rgb]{0.37,0.37,0.37}{#1}}
\newcommand{\SpecialStringTok}[1]{\textcolor[rgb]{0.13,0.47,0.30}{#1}}
\newcommand{\StringTok}[1]{\textcolor[rgb]{0.13,0.47,0.30}{#1}}
\newcommand{\VariableTok}[1]{\textcolor[rgb]{0.07,0.07,0.07}{#1}}
\newcommand{\VerbatimStringTok}[1]{\textcolor[rgb]{0.13,0.47,0.30}{#1}}
\newcommand{\WarningTok}[1]{\textcolor[rgb]{0.37,0.37,0.37}{\textit{#1}}}

\usepackage{longtable,booktabs,array}
\usepackage{calc} % for calculating minipage widths
% Correct order of tables after \paragraph or \subparagraph
\usepackage{etoolbox}
\makeatletter
\patchcmd\longtable{\par}{\if@noskipsec\mbox{}\fi\par}{}{}
\makeatother
% Allow footnotes in longtable head/foot
\IfFileExists{footnotehyper.sty}{\usepackage{footnotehyper}}{\usepackage{footnote}}
\makesavenoteenv{longtable}
\usepackage{graphicx}
\makeatletter
\newsavebox\pandoc@box
\newcommand*\pandocbounded[1]{% scales image to fit in text height/width
  \sbox\pandoc@box{#1}%
  \Gscale@div\@tempa{\textheight}{\dimexpr\ht\pandoc@box+\dp\pandoc@box\relax}%
  \Gscale@div\@tempb{\linewidth}{\wd\pandoc@box}%
  \ifdim\@tempb\p@<\@tempa\p@\let\@tempa\@tempb\fi% select the smaller of both
  \ifdim\@tempa\p@<\p@\scalebox{\@tempa}{\usebox\pandoc@box}%
  \else\usebox{\pandoc@box}%
  \fi%
}
% Set default figure placement to htbp
\def\fps@figure{htbp}
\makeatother





\setlength{\emergencystretch}{3em} % prevent overfull lines

\providecommand{\tightlist}{%
  \setlength{\itemsep}{0pt}\setlength{\parskip}{0pt}}



 


\KOMAoption{captions}{tableheading}
\makeatletter
\@ifpackageloaded{caption}{}{\usepackage{caption}}
\AtBeginDocument{%
\ifdefined\contentsname
  \renewcommand*\contentsname{Table of contents}
\else
  \newcommand\contentsname{Table of contents}
\fi
\ifdefined\listfigurename
  \renewcommand*\listfigurename{List of Figures}
\else
  \newcommand\listfigurename{List of Figures}
\fi
\ifdefined\listtablename
  \renewcommand*\listtablename{List of Tables}
\else
  \newcommand\listtablename{List of Tables}
\fi
\ifdefined\figurename
  \renewcommand*\figurename{Figure}
\else
  \newcommand\figurename{Figure}
\fi
\ifdefined\tablename
  \renewcommand*\tablename{Table}
\else
  \newcommand\tablename{Table}
\fi
}
\@ifpackageloaded{float}{}{\usepackage{float}}
\floatstyle{ruled}
\@ifundefined{c@chapter}{\newfloat{codelisting}{h}{lop}}{\newfloat{codelisting}{h}{lop}[chapter]}
\floatname{codelisting}{Listing}
\newcommand*\listoflistings{\listof{codelisting}{List of Listings}}
\makeatother
\makeatletter
\makeatother
\makeatletter
\@ifpackageloaded{caption}{}{\usepackage{caption}}
\@ifpackageloaded{subcaption}{}{\usepackage{subcaption}}
\makeatother
\usepackage{bookmark}
\IfFileExists{xurl.sty}{\usepackage{xurl}}{} % add URL line breaks if available
\urlstyle{same}
\hypersetup{
  pdftitle={Class 6: R Functions Lab},
  pdfauthor={Michael Romero (A18135877)},
  colorlinks=true,
  linkcolor={blue},
  filecolor={Maroon},
  citecolor={Blue},
  urlcolor={Blue},
  pdfcreator={LaTeX via pandoc}}


\title{Class 6: R Functions Lab}
\author{Michael Romero (A18135877)}
\date{}
\begin{document}
\maketitle


All functions in R have at least 3 things:

-A \textbf{name} we pick this and use it to call the function. - Input
\textbf{arguments}, there can be multiple comma separated inputs ot the
function - The \textbf{body}, lines of R code that do the work of the
function

Our first wee function:

\begin{Shaded}
\begin{Highlighting}[]
\NormalTok{add}\OtherTok{\textless{}{-}}\ControlFlowTok{function}\NormalTok{(x, }\AttributeTok{y=}\DecValTok{1}\NormalTok{) \{}
\NormalTok{  x}\SpecialCharTok{+}\NormalTok{y}
\NormalTok{\}}
\end{Highlighting}
\end{Shaded}

Lets test our function

\begin{Shaded}
\begin{Highlighting}[]
\FunctionTok{add}\NormalTok{(}\FunctionTok{c}\NormalTok{(}\DecValTok{1}\NormalTok{,}\DecValTok{2}\NormalTok{,}\DecValTok{3}\NormalTok{), }\AttributeTok{y=}\DecValTok{10}\NormalTok{)}
\end{Highlighting}
\end{Shaded}

\begin{verbatim}
[1] 11 12 13
\end{verbatim}

\begin{Shaded}
\begin{Highlighting}[]
\FunctionTok{add}\NormalTok{(}\DecValTok{10}\NormalTok{,}\DecValTok{100}\NormalTok{)}
\end{Highlighting}
\end{Shaded}

\begin{verbatim}
[1] 110
\end{verbatim}

\subsection{A second function}\label{a-second-function}

Let's try something more interesting. Make a sequence generation tool.

The \texttt{sample()} function could be useful here.

\begin{Shaded}
\begin{Highlighting}[]
\FunctionTok{sample}\NormalTok{(}\DecValTok{1}\SpecialCharTok{:}\DecValTok{10}\NormalTok{, }\AttributeTok{size=} \DecValTok{3}\NormalTok{)}
\end{Highlighting}
\end{Shaded}

\begin{verbatim}
[1] 2 3 8
\end{verbatim}

change this to work with the nucleotides A C G and T

\begin{Shaded}
\begin{Highlighting}[]
\NormalTok{n}\OtherTok{\textless{}{-}}\FunctionTok{c}\NormalTok{(}\StringTok{"A"}\NormalTok{,}\StringTok{"C"}\NormalTok{,}\StringTok{"T"}\NormalTok{,}\StringTok{"G"}\NormalTok{)}
\FunctionTok{sample}\NormalTok{(n, }\AttributeTok{size=}\DecValTok{5}\NormalTok{, }\AttributeTok{replace =} \ConstantTok{TRUE}\NormalTok{)}
\end{Highlighting}
\end{Shaded}

\begin{verbatim}
[1] "C" "A" "G" "T" "G"
\end{verbatim}

TUrn this snippet into a function that returns a user specified length
dna sequence. Let's call it \texttt{generate\_dna()}

\begin{Shaded}
\begin{Highlighting}[]
\NormalTok{generate\_dna}\OtherTok{\textless{}{-}}\ControlFlowTok{function}\NormalTok{(}\AttributeTok{len=}\DecValTok{10}\NormalTok{, }\AttributeTok{fasta=}\ConstantTok{FALSE}\NormalTok{) \{}
\NormalTok{  n}\OtherTok{\textless{}{-}}\FunctionTok{c}\NormalTok{(}\StringTok{"A"}\NormalTok{,}\StringTok{"C"}\NormalTok{,}\StringTok{"T"}\NormalTok{,}\StringTok{"G"}\NormalTok{)}
\NormalTok{  v}\OtherTok{\textless{}{-}} \FunctionTok{sample}\NormalTok{(n, }\AttributeTok{size=}\NormalTok{len, }\AttributeTok{replace =} \ConstantTok{TRUE}\NormalTok{)}
  
  \CommentTok{\# Make a single element vector}
\NormalTok{  s}\OtherTok{\textless{}{-}}\FunctionTok{paste}\NormalTok{(v, }\AttributeTok{collapse=}\StringTok{""}\NormalTok{)}
  
  \FunctionTok{cat}\NormalTok{(}\StringTok{"Well done you!"}\NormalTok{)}
  
  \ControlFlowTok{if}\NormalTok{(fasta) \{}
 \FunctionTok{return}\NormalTok{(s)    }
\NormalTok{  \}}\ControlFlowTok{else}\NormalTok{\{}
    \FunctionTok{return}\NormalTok{(v)}
\NormalTok{  \}}
\NormalTok{\}}
\end{Highlighting}
\end{Shaded}

\begin{Shaded}
\begin{Highlighting}[]
\FunctionTok{generate\_dna}\NormalTok{(}\DecValTok{5}\NormalTok{)}
\end{Highlighting}
\end{Shaded}

\begin{verbatim}
Well done you!
\end{verbatim}

\begin{verbatim}
[1] "T" "G" "A" "A" "A"
\end{verbatim}

\begin{Shaded}
\begin{Highlighting}[]
\NormalTok{s}\OtherTok{\textless{}{-}}\FunctionTok{generate\_dna}\NormalTok{(}\DecValTok{15}\NormalTok{)}
\end{Highlighting}
\end{Shaded}

\begin{verbatim}
Well done you!
\end{verbatim}

\begin{Shaded}
\begin{Highlighting}[]
\NormalTok{s}
\end{Highlighting}
\end{Shaded}

\begin{verbatim}
 [1] "A" "G" "C" "A" "G" "A" "G" "C" "C" "A" "C" "A" "C" "T" "G"
\end{verbatim}

I want the option to return a single element character vector with my
sequence all together like this: ``GGAGTAC''

\begin{Shaded}
\begin{Highlighting}[]
\FunctionTok{generate\_dna}\NormalTok{(}\DecValTok{10}\NormalTok{, }\AttributeTok{fasta=}\ConstantTok{FALSE}\NormalTok{)}
\end{Highlighting}
\end{Shaded}

\begin{verbatim}
Well done you!
\end{verbatim}

\begin{verbatim}
 [1] "G" "T" "A" "C" "T" "G" "A" "C" "G" "C"
\end{verbatim}

\begin{Shaded}
\begin{Highlighting}[]
\FunctionTok{generate\_dna}\NormalTok{(}\DecValTok{10}\NormalTok{, }\AttributeTok{fasta=}\ConstantTok{TRUE}\NormalTok{)}
\end{Highlighting}
\end{Shaded}

\begin{verbatim}
Well done you!
\end{verbatim}

\begin{verbatim}
[1] "GTCAACCGCT"
\end{verbatim}

\subsection{A more advanced example}\label{a-more-advanced-example}

Make a third function that generates protein sequence of a user
specified length and format.

\begin{Shaded}
\begin{Highlighting}[]
\NormalTok{generate\_dna}\OtherTok{\textless{}{-}}\ControlFlowTok{function}\NormalTok{(}\AttributeTok{size=}\DecValTok{20}\NormalTok{, }\AttributeTok{fasta=}\ConstantTok{FALSE}\NormalTok{) \{}
\NormalTok{  n}\OtherTok{\textless{}{-}}\FunctionTok{c}\NormalTok{(}\StringTok{"A"}\NormalTok{,}\StringTok{"C"}\NormalTok{,}\StringTok{"T"}\NormalTok{,}\StringTok{"G"}\NormalTok{)}
\NormalTok{  v}\OtherTok{\textless{}{-}} \FunctionTok{sample}\NormalTok{(n, }\AttributeTok{size=}\NormalTok{size, }\AttributeTok{replace =} \ConstantTok{TRUE}\NormalTok{)}
  
  \CommentTok{\# Make a single element vector}
\NormalTok{  s}\OtherTok{\textless{}{-}}\FunctionTok{paste}\NormalTok{(v, }\AttributeTok{collapse=}\StringTok{""}\NormalTok{)}
  
  \FunctionTok{cat}\NormalTok{(}\StringTok{"Well done you!"}\NormalTok{)}
  
  \ControlFlowTok{if}\NormalTok{(fasta) \{}
 \FunctionTok{return}\NormalTok{(s)    }
\NormalTok{  \}}\ControlFlowTok{else}\NormalTok{\{}
    \FunctionTok{return}\NormalTok{(v)}
\NormalTok{  \}}
\NormalTok{\}}
\end{Highlighting}
\end{Shaded}

\begin{Shaded}
\begin{Highlighting}[]
\FunctionTok{generate\_dna}\NormalTok{(}\DecValTok{6}\NormalTok{)}
\end{Highlighting}
\end{Shaded}

\begin{verbatim}
Well done you!
\end{verbatim}

\begin{verbatim}
[1] "T" "G" "A" "C" "A" "A"
\end{verbatim}

\begin{Shaded}
\begin{Highlighting}[]
\NormalTok{generate\_protein}\OtherTok{\textless{}{-}}\ControlFlowTok{function}\NormalTok{(}\AttributeTok{size=}\DecValTok{15}\NormalTok{, }\AttributeTok{fasta=}\ConstantTok{TRUE}\NormalTok{) \{}
\NormalTok{  aa}\OtherTok{\textless{}{-}}\FunctionTok{c}\NormalTok{(}\StringTok{"A"}\NormalTok{,}\StringTok{"R"}\NormalTok{,}\StringTok{"N"}\NormalTok{,}\StringTok{"D"}\NormalTok{,}\StringTok{"C"}\NormalTok{,}\StringTok{"Q"}\NormalTok{,}\StringTok{"E"}\NormalTok{,}\StringTok{"G"}\NormalTok{,}\StringTok{"H"}\NormalTok{,}\StringTok{"I"}\NormalTok{,}\StringTok{"L"}\NormalTok{,}\StringTok{"K"}\NormalTok{,}\StringTok{"M"}\NormalTok{,}\StringTok{"F"}\NormalTok{,}\StringTok{"P"}\NormalTok{,}\StringTok{"S"}\NormalTok{,}\StringTok{"T"}\NormalTok{,}\StringTok{"W"}\NormalTok{,}\StringTok{"Y"}\NormalTok{,}\StringTok{"V"}\NormalTok{)}
\NormalTok{  seq}\OtherTok{\textless{}{-}}\FunctionTok{sample}\NormalTok{(aa, }\AttributeTok{size=}\NormalTok{size, }\AttributeTok{replace=}\ConstantTok{TRUE}\NormalTok{)}
  
  \ControlFlowTok{if}\NormalTok{ (fasta) \{}
    \FunctionTok{return}\NormalTok{(}\FunctionTok{paste}\NormalTok{(seq, }\AttributeTok{collapse=}\StringTok{""}\NormalTok{))}
\NormalTok{  \} }\ControlFlowTok{else}\NormalTok{\{}
    \FunctionTok{return}\NormalTok{(seq)}
\NormalTok{    \}}
\NormalTok{\}}
\end{Highlighting}
\end{Shaded}

\begin{Shaded}
\begin{Highlighting}[]
\FunctionTok{generate\_protein}\NormalTok{(}\DecValTok{5}\NormalTok{)}
\end{Highlighting}
\end{Shaded}

\begin{verbatim}
[1] "MCALC"
\end{verbatim}

\begin{quote}
Q. generate random protein sequences between lengths 5 and 12 amino
acids.
\end{quote}

One approach is to do this by brute force calling our function for each
length 5 to 12.

Another approach is to write a \texttt{for()} loop to itterate over the
input valued 5 to 12

A very useful third R specific approach is to use the \texttt{sapply()}
function.

\begin{Shaded}
\begin{Highlighting}[]
\FunctionTok{sapply}\NormalTok{(}\DecValTok{5}\SpecialCharTok{:}\DecValTok{12}\NormalTok{, generate\_protein)}
\end{Highlighting}
\end{Shaded}

\begin{verbatim}
[1] "MFYDP"        "KVGSHV"       "KRLVYQE"      "QYCSDELS"     "HKVKVVRSS"   
[6] "EKCTPFCPQY"   "WTCTSWSILFY"  "DHQHPQWTMILQ"
\end{verbatim}

\begin{quote}
\textbf{Key-Point}: Writing functions in R is doable but not the easiest
thing. Starting with a working snippet of code and then using LLM tools
to imporve and generalize your function code is a productive approach.
\end{quote}




\end{document}
